\documentclass[journal]{IEEEtran}

\usepackage{cite}
\usepackage[pdftex]{graphicx}
\usepackage{amsmath} 
\interdisplaylinepenalty=2500
\usepackage{algorithmic}
\usepackage{url}


\begin{document}
\title{Introduction to Statistical Computing in Scala - an Implementation of the K-Nearest Neighbors classifier}
\author{Roxana~Tesileanu,~\IEEEmembership{Research Assistant, ~National Institute of Forest Research and Management}}

\markboth{IEEE Access, ~Vol.~xx, No.~xx, October~2017}
{Tesileanu: Introduction to Statistical Computing in Scala - an Implementation fo the K-Nearest Neighbors classifier}

\maketitle

\begin{abstract}

Statistical computing in ecology evolves at a high speed, mainly because researchers have recognized the advantage of being able to design their algorithms according to their needs. The present paper introduces the implementation in Scala of the k-Nearest Neighbors (kNN) classifier based on Euclidean distances, which can be applied also on small datasets, a situation commonly encountered in ecological research. 

\end{abstract} 

