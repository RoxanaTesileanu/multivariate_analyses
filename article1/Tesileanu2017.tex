\documentclass[journal]{IEEEtran}

\usepackage{cite}
\usepackage[pdftex]{graphicx}
\usepackage{amsmath} 
\interdisplaylinepenalty=2500
\usepackage{algorithmic}
\usepackage{url}


\begin{document}
\title{Introduction to Statistical Computing in Scala - an Implementation of the K-Nearest Neighbors classifier}
\author{Roxana~Tesileanu,~\IEEEmembership{Research Assistant,~National Institute of Forest Research and Management (INCDS), Romania}}

\markboth{IEEE Access, ~Vol.~xx, No.~xx, October~2017}
{Tesileanu: Introduction to Statistical Computing in Scala - an Implementation fo the K-Nearest Neighbors classifier}

\maketitle

\begin{abstract}

Statistical computing in ecology evolves at a high speed, mainly because researchers have recognized the advantage of being able to design their algorithms according to their needs. The present paper introduces the implementation in Scala of the k-Nearest Neighbors (kNN) classifier based on Euclidean distances, which can be applied also on small datasets, a situation commonly encountered in ecological research. 

\end{abstract} 

\begin{IEEEkeywords}
machine learning in ecology, k-Nearest Neighbors classification, Scala
\end{IEEEkeywords}

\section{Introduction}

One of the drivers of the machine-learning progress is the great amount of data born within and gathered by networked and mobile computing systems which necessitates further processing in order to gain insights into the specific fields from which it originates (Jordan and Mitchell 2015, p. 256). 
The "Big Data phenomenon" is real but, unfortunately, it can not be generalized to all research areas, especially some areas of ecological research which investigate systems which are by nature data-poor and will probably remain as such unless cost-intensive data collection projects are being proposed and financed. This by no means implies that such areas cannot take advantage of the progress of machine-learning and take the best out of the existing datasets.        
On the contrary, machine-learning, being a study field which "sits at the crossroads of computer science, statistics and a variety of other disciplines concerned with automatic improvement over time, and inference and decision-making under uncertainty" (Jordan and Mitchell 2015, p. 256), is welcomed in real-world environmental decision-making facing the need of proposing courses of action under the cloud of uncertainties (Vatn and Bromley, 1994).   








\section{Conclusion}
\appendices
\section{Source file 1}

......

\section{Source file 2}

......

\section*{Acknowledgment}
The author would like to thank Jeff Druce and Mike Reposa for very useful learning tips.
%\begin{thebibliography}{1}
%\bibitem{ item: 1}
%...
%\end{thebibliography}


\begin {IEEEbiographynophoto}{Roxana Tesileanu}
My bio.
\end{IEEEbiographynophoto}

\end{document}


