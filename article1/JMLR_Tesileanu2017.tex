\documentclass[twoside, 11p]{article}

\usepackage{jmlr2e}
\usepackage{cite} %bibliographystyle plannat

\begin{document}
\title{Introduction to Statistical Computing in Scala - an Implementation of the K-Nearest Neighbors classifier}

\author{\name Roxana Tesileanu \email roxana.te@web.de \\
	\addr Environmental Statistics and Bioinformatics \\
	National Institute of Forest Research and Management (INCDS)\\
	Brasov, Romania} 
\editor{Kevin Murphy and Bernhard Sch{\"o}lkopf}

\maketitle

\begin{abstract}%
 Statistical computing in ecology evolves at a high speed, mainly because researchers have recognized the advantage of being able to design their algorithms according to their needs.
 The present paper introduces the implementation in Scala language of the k-Nearest Neighbors (kNN) classifier, which can be applied also on small datasets, a situation commonly encountered in ecological research, and discusses its possible role within a more complex learning pipeline for classification tasks. 
\end{abstract}

\begin{keywords}
learning pipeline for classification, machine learning in ecology, diffusion processes,  k-Nearest Neighbors classification, Scala language, Simple Build Tool (SBT), multivariate analyses 
\end{keywords}

\section{Introduction}
   



