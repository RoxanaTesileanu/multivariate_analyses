\documentclass {article}

\title {Geospatial analysis in Scala}
\date {October 2017}

\usepackage {amsmath}
\usepackage {graphics}
\usepackage {verbatim}
\usepackage {cite}
\usepackage {booktabs}
\usepackage {float}
\usepackage [titletoc, toc, title] {appendix}
\usepackage {hyperref}


\author {Roxana Tesileanu \\
\\
\href{mailto: roxana.te@web.de}{roxana.te@web.de} \\
INCDS, Romania}

\begin {document}
	\maketitle

\tableofcontents

\section {Introduction}

The aim of the present paper is to investigate the use of some of the existing libraries for geospatial analysis available in Scala, the Geospatial Data Abstraction Library (GDAL) and Geotrellis, for performing the main geospatial analysis tasks: manipulating vector and raster data (geoprocessing) and geostatistics.
The former task will be approched using GDAL and the later using Geotrellis.  

\section {Geoprocessing using GDAL}

Using a programming language for geospatial analysis allows you to customize your analyses instead of being limited to what the software user interface allows. This is one of the most important advantages of open source software \cite{garrard_geoprocessing_2016}. The GDAL library is one of the open source libraries used in this work. It was written in C and C++ and has bindings for several languages (Java, Perl and Python). \\     
In order to use GDAL, you need to install it on your machine, and for its import in Scala you need to install its Java bindings along with it.  
For installation details you can look at the GDAL homepage \href{http://www.gdal.org/}{http://www.gdal.org/}, download GDAL and follow the instructions for building from source, which might not be an easy task, depending on your operating system. 
Thanks to the efforts of the UbuntuGIS team (\href{https://wiki.ubuntu.com/UbuntuGIS}{https://wiki.ubuntu.com/UbuntuGIS}), on Ubuntu, the installation procedure of GDAL and its bindings is done rapidly. Firstly, you need to add the ubuntugis PPA, which offers the official stable UbuntuGIS packages, to your system (\href{https://launchpad.net/~ubuntugis/+archive/ubuntu/ppa}{https://launchpad.net/~ubuntugis/+archive/ubuntu/ppa}). This is done with the commands: \\
\\
sudo add-apt-repository ppa:ubuntugis/ppa \\
sudo apt-get update.\\
\\
Next, you install GDAL on your machine with the commands \cite{safavi_installing_2015} \cite{noauthor_ubuntugis_nodate} (\href{http://www.sarasafavi.com/installing-gdalogr-on-ubuntu.html}{http://www.saras\\ afavi.com/installing-gdalogr-on-ubuntu.html}, \href{https://packages.ubuntu.com/source/trusty/gdal}{https://packages.ubuntu.com/sou\\ rce/trusty/gdal}):\\
\\
sudo apt-get install libproj-dev, gdal-bin, libgdal-dev, libgdal-doc \\
sudo apt-get update.\\
\\
Finally, you add the Java bindings to your GDAL package (\href{https://launchpad.net/ubuntu/+source/proj}{https://launchpad.ne\\ t/ubuntu/+source/proj}):\\
\\
sudo apt-get install libgdal-java, libproj-java.    \\
\\
In order to import GDAL in Scala, you have to add its jar to the project's classpath. An easy way of managing dependencies of a Scala project is to use SBT (for further details see \cite{tesileanu_using_2017}). In this way you can take advantage of the most convenient way to place the gdal jar to the project's classpath, namely, to place a copy of it into the lib directory of the Scala project, now that the actual installation has already taken place. \\ 
\\
The following subsections will offer a background in geoprocessing, starting with manipulating vector data (reading and writing files of different vector data formats and performing overlay and proximity analyses), and continuing with manipulating raster data (reading and writing files of different raster data formats, resizing pixels, performing moving window analyses and map algebra).    

\subsection {Types of spatial data}

Spatial data are divided in two categories: vector data and raster data. Vector data provide information about distinct features in space, i.e. different distinct items of interest, and are made up of points, lines and polygons \cite{garrard_geoprocessing_2016}. The features of interest could be for example:
\begin{itemize}
\item roads, rivers, road networks, hidrological networks, country boundaries, city boundaries as examples of features represented by lines,
\item  mountain peaks, volcano peaks, weather stations, restaurants, as examples of features represented by points, and 
\item lakes, oceans, ownership status as examples of features represented by polygons.     
\end{itemize}
Features have attributes attached to them such as the name of the individual observations (for example the wheather stations's name) and other recorded variables (like for example different concentrations of air pollutants, temperature or wind regime for each individual weather station). As it can be noticed, the multiple attributes which can be attached to features, can be of different types, and they actually represent different types of recorded variables (they might be dicrete or continuous numerical variables or categorical variables). \\ 
\\
On the other hand, raster data provide information about characteristics of interest which take the form of a continuum like gradients, with no distinct boundaries. They are represented as two- or three-dimensional arrays of data values which form grids of values \cite{garrard_geoprocessing_2016}.
Because they can cope well with gradients, they capture local variation more easily than vector geometries, and are used in digital elevation models (DEMs). Also because the data source is pixel-based (e.g. aerial photos, satellite imagery) they can be used in vegetation mapping.     

\subsection {Reading vector data}

The main objective of vector data analysis is to investigate relationships between features, by overlapping them on another or measuring distances between them \cite{garrard_geoprocessing_2016}. 
A typical example for vector analyses is the investigation of GPS-collared wildlife to see the direction of travel, distances covered and how they interact with man-made features like roads \cite{garrard_geoprocessing_2016}. \\
In order to perform such vector-based analyses, we need to be able to read, edit and write vector data. This kind of functionality is offered by the OGR Simple Features Library for geoprocessing vector data, which is included in GDAL. \\
\\
At this point it is noted that the Scala code relating to using the GDAL functionality introduced in this document has its origins in the Python code written by Chris Garrard in her book "Geoprocessing with Python" (2016). The main reason for the transition towards using Scala for geospatial analysis is the use of Scala's functional nature for the further processing of geodata, by using higher-order functions. \\
\\
There are many different types of vector data formats. Among the most widely used ones are: the ESRI shapefile, the GeoJSON file, or the SpatiaLite or PostGIS databases. 
The ESRI shapefile format requires a minimum of three binary files, each of which serves a different purpose: geometry information is stored in .shp and .shx files, and attribute values are stored in a .dbf file. You need to make sure they are all grouped in the same folder, because they work together \cite{garrard_geoprocessing_2016}. The GeoJSON format is used mainly for web-mapping applications and is a plain text file which can be easily examined. The GeoJSON format consists of a single file. Vector data can also be stored in relational databases with spatial extensions. The most widely used spatial extensions are SpatiaLite (for SQLite databases) and PostGIS (for PostgreSQL). 
You can check other vector data formats supported by GDAL at \href{http://www.gdal.org/ogr_formats.html}{http://www.gdal.org/ogr\underline{\space}formats.html}. \\ 
\\
The OGR package of GDAL contains the classes used for geoprocessing vector data. The OGR Java Application Programming Interface (API) \cite{noauthor_gdal_nodate} (\href{http://gdal.org/java/}{http://gdal.org/java/}) lists them all.
 Among them there are the classes: Driver, DataSource, Layer, Geometry and Feature. In order to handle vector geodata with OGR, we need to understand how the geospatial information is organized in OGR. The spatial vector data is stored in a data source (for example a shapefile, a GeoJSON file, or a SpatiaLite or PostGIS database). This data source object can have one or more layers, one for each dataset contained in the data source. Many vector formats, such as shapefiles can only contain one dataset, thus have one layer, but others like SpatiaLite can contain multiple datasets, thus have multiple layers.                


\subsection {Georeferencing data}      

\subsection {Overlay analyses}

\subsection {Proximity analyses}

\subsection {Writing vector data}

\subsection {Reading raster data}

\subsection {Pixels resizing}

\subsection {Moving window analyses}

\subsection {Map algebra}



\section {Geostatistics using Geotrellis}

%\section* {References}
\bibliography {MyLibrary}
\bibliographystyle {plain}


\end {document}
