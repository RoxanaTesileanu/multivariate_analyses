\documentclass {article}

\title {Geospatial analysis in Scala}
\date {October 2017}

\usepackage {amsmath}
\usepackage {graphics}
\usepackage {verbatim}
\usepackage {cite}
\usepackage {booktabs}
\usepackage {float}
\usepackage [titletoc, toc, title] {appendix}

\author {Roxana Tesileanu \\
\\
roxana.te@web.de \\
INCDS, Romania}

\begin {document}
	\maketitle

\tableofcontents

\section {Introduction}

The aim of the present paper is to investigate the use of some of the existing libraries for geospatial analysis available in Scala, the Geospatial Data Abstraction Library (GDAL) and Geotrellis, for performing the main geospatial analysis tasks: manipulating vector and raster data (geoprocessing) and geostatistics.
The former task will be approched using GDAL and the later using Geotrellis.  

\section {Geoprocessing using GDAL}

Using a programming language for geospatial analysis allows you to customize your analyses instead of being limited to what the software user interface allows. It also allows the processing of more datasets at once \cite{garrard_geoprocessing_2016}.   




\section {Geostatistics using Geotrellis}

\section* {References}
\bibliography {MyLibrary}
\bibliographystyle {plain}


\end {document}
